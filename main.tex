\documentclass{article}

\usepackage[english]{babel}
\usepackage{multicol}
\setlength{\columnsep}{1cm}

\usepackage[a4paper,top=2cm,bottom=2cm,left=2cm,right=2cm,marginparwidth=1.75cm]{geometry}

\usepackage{amsmath}
\usepackage{siunitx}
\usepackage{wrapfig}
\usepackage{float}
\usepackage{graphicx}
\usepackage{subcaption}
\usepackage[colorlinks=true, allcolors=blue]{hyperref}
\usepackage{xcolor}
\usepackage{lipsum}
\usepackage{mathtools}
\usepackage{listings}
\usepackage{xcolor}





\title{Aandrijftechniek maan casus}

\author{Jelmer Hemstra, 1810225}

\begin{document}
    \maketitle

    \begin{abstract}
        In dit document wordt de casus van de aandrijftechniek van de maanlander behandeld. Hierbij wordt gekeken naar de verschillende aandrijftechnieken en de voor- en nadelen van deze technieken.

    \end{abstract}


\section{Inleiding}
De maanlander is een voertuig dat gebruikt wordt om mensen en goederen van de maan naar de aarde te vervoeren. 
De maanlander moet in staat zijn om te landen op de maan en weer op te stijgen. 
Om dit te kunnen doen is een goede aandrijftechniek nodig. 
In dit document wordt gekeken naar de verschillende aandrijftechnieken en de voor- en nadelen van deze technieken.

\section{analyze}
- vraagstellingen 
- Eisen
- gegevens



\section{Lasten}
- lasten berekening laten zien en uitleggen


- lasten tabel laten zien en uitleggen






\section{motor}
- motor kromme berkennenn

- motor kromme laten zien en uitleggen

- rendement berekenenn


\section{Conclusion}

iets zeggen over redement enzo



\end{document}

