\documentclass{article}

\usepackage[dutch]{babel}
\usepackage{multicol}
\setlength{\columnsep}{1cm}

\usepackage[a4paper,top=2cm,bottom=2cm,left=2cm,right=2cm,marginparwidth=1.75cm]{geometry}

\usepackage{amsmath}
\usepackage{siunitx}
\usepackage{wrapfig}
\usepackage{float}
\usepackage{graphicx}
\usepackage{subcaption}
\usepackage[colorlinks=true, allcolors=blue]{hyperref}
\usepackage{xcolor}
\usepackage{lipsum}
\usepackage{mathtools}
\usepackage{listings}
\usepackage{xcolor}
\usepackage{pdfpages}
\graphicspath{{./photos/}}




\title{Aandrijftechniek Euro Moon Rover Casus}

\author{Jelmer Hemstra, 1810225, Flint Wardenaar, 1771881}


\begin{document}
\maketitle
\begin{figure}[h]
    \centering
    \includegraphics[width=1\textwidth]{Maanlanderfixed.png}
    \caption{Moonrover gemaakt door DALL-E}
    \label{fig:moonrover2}
\end{figure}
\newpage
\tableofcontents
\newpage
\begin{abstract}
    Dit document is een raportage van de casus Euro Moon Rover. 
    Dit hoort bij de opdracht van het vak Aandrijftechniek.
    De opdracht is om een motor en gearbox te selecteren voor de Euro Moon Rover.
    Er zijn hiervoor een aantal eisen opgesteld die in dit document worden behandeld.
    Er wordt gekeken naar de statische en dynamische lasten die de motor moet verdragen.
    Ook wordt er gekeken naar de meest voorkomende last en hoe deze het beste kan worden opgelost.
    Uit dit onderzoek is het advies gekomen om de RE25 118745 motor te gebruiken in combinatie met de GP 26A 406757 gearbox.
\end{abstract}


\section{Inleiding}
    In dit verslag gaan we onderzoeken welke motor en gearbox het beste is voor de Euro Moon Rover.
    We beginnen met het analyseren van de eisen en gegevens die we hebben gekregen.
    Daarna gaan we onderzoeken welke lasten de motor moet verdragen en welke motor en gearbox het beste is voor deze lasten.
    In de resultaten bespreken we de uitkomsten van het onderzoek en in het advies geven we aan welke motor en gearbox het beste is voor de Euro Moon Rover.


    \section{Analyse}
        \subsection{Onderzoeksvragen}
            Hoofdvraag: \newline
            Welke combinatie van motor en overbrenging, 
            van de familie ``RE25 1187xx'' en ``Planetary Gearhead GP xx xx'' respectievelijk, 
            is het meest geschikt als aandrijving van de ``Euro Moon Rover``?
            \newline \newline
            Deelvragen: 
            \begin{itemize}
                \item Wat is de statische last?
                \item Wat is de dynamische last?
                \item Wat is de meest voorkomende last?
            \end{itemize}

        \subsection{Eisen}
            Uit de opdrachtsbeschrijving zijn de volgende eisen gehaald:
            \begin{enumerate}
                \item De rover moet een helling van $20^{\circ}$ op en af kunnen rijden.
                \item De rover moet kunnen versnellen met $0.7[m/s^2]$ en vertragen met $0.5[m/s^2]$.
                \item De rover moet een topsnelheid hebben van ninstens $2.1[m/s]$.
                \item De motor moet deel uitmaken van de ``RE25 1187xx'' familie en heeft een diameter van 25mm.
                \item De overbrenging moet deel uitmaken van de ``Planetary Gearhead GP xx xx'' familie.
            \end{enumerate}
        
        \subsection{Gegevens}
            Uit de opdracht zijn de volgende gegevens gehaald:
            \begin{itemize}
                \item De massa van de de rover: $m = 6[kg]$
                \item De valversnelling op de maan: $g_m = 1.62[m/s^2]$
                \item De rolweerstandscoëfficiënt: $\mu_r = 0.1$
                \item De straal van de wielen: $r = 0.075[m]$
                \item De massatraagheidvan de wielen: $J = 0.0021[kg \cdot m^2]$
            \end{itemize}
            Ook zijn de eisen genoteerd als gegevens:
            \begin{itemize}
                \item De maximale helling: $\theta_{max} = 20^\circ$
                \item De maximale versnelling: $a_{max} = 0.7[m/s^2]$
                \item De maximale snelheid: $v_{max} = 2.1[m/s]$
            \end{itemize}

    \section{Onderzoek}
        Om te bepalen welk type motor het beste is voor de toepassing wordt er vooral gekeken naar de last die de motor moet verdragen.
        De last is opgedeeld in statische en dynamische last. 
        De statische last is de last die de motor moet verdragen als de rover een vaste snelheid heeft.
        De dynamische last is de last die de motor moet verdragen als de rover versnelt of vertraagt.

        \begin{equation}
            T_{tot} = T_{stat} + T_{dyn} 
        \end{equation}


        % Beschrijf de aanpak en methoden die gebruikt zijn voor het onderzoek 
        % en de evaluatie van de motoren en transmissiesystemen.
        % Duidelijke formulering van de vraag waarop
        % door middel van analyse/onderzoek een 
        % antwoord wordt gezocht. (max 6 A4’tjes)



        \subsection{Statische last}
            Er zijn twee onderdelen in de statische last, namelijk de zwaartekracht en de rolweerstand. 

            \subsubsection*{Zwaartekracht}
                Met de zwaartekracht wordt de kracht bedoeld die resulteerd uit de kracht die de rover omlaag duwt 
                en de normaalkracht van het oppervlakte. 
                Deze resulterende kracht is ervoor verantwoordelijk dat de rover de helling af ``wil'' rollen.
                Deze is te berekenen met de formule:
                
                \begin{equation}
                    F_{z} = m \cdot g \cdot \sin(\theta) 
                \end{equation}

                \begin{itemize}
                    \item $F_{z}$ is last die de zwaartekracht veroorzaakt in $[N]$
                    \item $m$ is de massa van de rover in $[kg]$
                    \item $g$ is de zwaartekracht in $[m/s^2]$
                    \item $\theta$ is de hoek van de helling in $[rad]$
                \end{itemize}
            
            \subsubsection*{Rolweerstand}
                De rolweerstand is de kracht die het rollen tegenwerkt en is een gevolg van de frictie tussen de wielen en de grond.
                Deze kracht is te berekenen met de formule:

                \begin{equation}
                    F_{rw} =  u_{r} \cdot m \cdot g \cdot \cos(\theta)
                \end{equation}

                \begin{itemize}
                    \item $F_{rw}$ is de last die de rolweerstand veroorzaakt in $[N]$
                    \item $u_{r}$ is de rolweerstandscoëfficiënt
                    \item $g$ is de zwaartekracht in $[m/s^2]$
                    \item $\theta$ is de hoek van de helling in $[rad]$
                \end{itemize}

            \subsubsection*{Totaal statische last}
                De totale statische last is dan de som van de zwaartekracht en de rolweerstand:

                \begin{equation}
                    F_{stat} = F_{z} + F_{rw}
                \end{equation}

                Om de motor te selecteren moet er gekeken worden naar de koppel. 
                Om de koppel te berekenen per motor moet de volgende formule gebruikt worden:

                \begin{equation}
                    T_{stat} = \frac{F_{stat} \cdot r}{4}
                \end{equation}

                Let hierbij op dat de deze last \textbf{per wiel} geldt.



        \subsection{Dynamische last}
            De dynamische last volgt uit twee onderdelen: de massa van de rover en de massatraagheid van de wielen.
            
            \subsubsection*{Massa}
                De last die volgt uit de massa van de rover is te berekenen met de formule:

                \begin{equation}
                    T_{m} = m \cdot a \cdot r
                \end{equation}

                \begin{itemize}
                    \item $T_{m}$ is het koppel die nodig is om de rover te versnellen $[Nm]$
                    \item $m$ is de massa van de rover in $[kg]$
                    \item $a$ is de versnelling van de rover in $[m/s^2]$
                    \item $r$ is de straal van het wiel in $[m]$
                \end{itemize}
                
            \subsubsection*{Massatraagheid}
                    

                De massatraagheid \textbf{per wiel} is te berekenen met de formule:

                \begin{equation}
                    T_{J} = \frac{J \cdot a}{r}
                \end{equation}

                \begin{itemize}
                    \item $T_{J}$ is het koppel dat nodig is voor de hoekversnelling van het wiel, in $[Nm]$
                    \item $J$ is het traagheidsmoment van de wielen, in $[kg \cdot m^2]$
                    \item $a$ is de versnelling van de rover, in $[m/s^2]$
                    \item $r$ is de straal van het wiel, in $[m]$
                \end{itemize}
                
            \subsubsection*{Totaal dynamische last}
                De totale dynamische last koppel \textbf{per wiel} is beschreven met de formule:
                \begin{equation}
                    T_{dyn} = \frac{T_{m}}{4} + T_{J}
                \end{equation}


        \subsection{Meest voorkomende last}
            Omdat de rover op zonne energie zal gaan rijden, is het belangrijk dat de energie efficiënt gebruikt wordt.
            Door te onderzoeken welke last het meeste voorkomt, 
            kan de motor en overbrenging zo worden gekozen om het meest efficiënt te werken voor die last. In lastberekeningen zijn maar drie variabelen: 
            snelheid, versnelling en hellingshoek. 
            \subsubsection*{Dynamiek}
                We gaan ervan uit dat de rover altijd probeert om zijn maximale snelheid te rijden.
                De rover zal op zijn langst 7 seconden in versnelling zijn. 4 seconden vertragen en dan 3 seconden versnellen. 
                We maken de aanname dat de gemiddelde reistijd van de rover vele malen groter zal zijn dan die 7 seconden.
                Om deze reden optimaliseren we de efficiëntie enkel op de statische last.

            \subsubsection*{Terein}
                Om een inschatting te maken over het terein op de maan is er een schriftelijk onderzoek gedaan.
                Uit dit onderzoek is een publicatie van NASA gevonden uit 1969 (nasa sp-8023, 1969), waarin de eigenschappen van het maanlandschap worden beschreven.
                Dit onderzoek is te vinden in Bijlage \ref{Bijlage C}.
                In de grafiek in figuur \ref{fig:maanhelling} is te zien hoe groot de gemiddelde helling is op de maan.
                Door te kijken naar de kleinste hellinglengte en het nominale landschap, zien we dat de gemiddelde helling zo'n $8.5^\circ$ is.

                \begin{figure}[h]
                    \includegraphics[width=\textwidth]{variation_of_mean_lunar_slope}
                    \caption{Variatie gemiddelde maanhelling}
                    \label{fig:maanhelling}
                    \centering
                \end{figure}
                
    \pagebreak
    \section{Resultaten}
        In dit hoofdstuk worden de resultaten van het onderzoek gepresenteerd.

    \subsection{last}
    De statische en dynamische lasten zijn berekend voor verschillende hellingen. 
    Deze berekeningen zijn gedaan met formule 5 en 8.
    De hellingen die we hebben gekozen om uit te rekenen zijn -20, -8,5, 0, 8,5 en 20 graden.
    De reden hiervoor is toegelicht in het onderzoek.
    Uit deze berekeningen zijn de volgende resultaten gekomen: 

        \begin{table}[h]
            \centering
            \begin{tabular}{|c|c|c|c|c|c|}
            \hline
            Helling & $-20 ^\circ$ & $-8.5 ^\circ$ & $0 ^\circ$ & $8.5 ^\circ$ & $20 ^\circ$ \\ \hline
            Statisch [mNm]  & -45.20   & -8.91   & 18.22   & 44.96  & 79.45   \\ \hline
            Dynamisch [mNm] & 98.33    & 98.33   & 98.33   & 98.33  & 98.33  \\ \hline
            Totaal [mNm] & 53.13  & 89.43   & 116.57  & 143.31  & 177.81  \\ \hline
            \end{tabular}
            \caption{Helling - Koppel}
            \label{tab}
        \end{table}
    
    Deze tabel laat zien dat de grootste last die de motor moet verdragen is bij  een versnelling van 0.7 $[m/s^2]$ op een helling van 20 graden.
        
    \subsection{Motor}
    De motor is gekozen aan de hand van de tabel 1 in hoofdstuk 3.1. 
    Ook is er gekeken naar de gearbox daar geldt namenlijk dat hoe minder vertanding de gearbox hoeft te doen hoe efficienter deze is. 
    Hierdoor is er gekozen voor een so traag mogelijke motor binnen de RE25 1187 serie.
    De motor die gekozen is is de RE25 118745 de datasheet van deze motor staat in bijlage \ref{Bijlage A}.
    Hiervan staat de datasheet in de bijlage.
    Deze motor heeft een nominale snelheid van 3710 [rpm], een maximale efficientie van 90 procent en een nominale koppel van 28.7 [mNm].
    De no-load snelheid is 4790 [rpm].
    

    \subsection{gearbox}
    De gearbox is gekozen aan de hand van de motor. 
    In de datasheet van de motor worden namenlijk een aantal gearboxes aangeraden.
    Een daarvan is degende die wij gekozen hebben, namenlijk de GP 26A 406757. 
    De datasheet van deze gearbox staat in bijlage \ref{Bijlage B}.
    Deze gearbox heeft een vertanding van 5.2:1. dit zorgt voor een Nominale snelheid van 710 [rpm] en een nominale koppel van 150 [mNm]. 
    Dit is meer dan genoeg koppel om met een constante snelheid van 2.1 [m/s] te rijden op een helling van 20 graden.

\section{Advies}
    Het advies wat blijkt uit dit verslag is om de RE25 118745 motor te gebruiken in combinatie met de GP 26A 406757 gearbox. 
    Deze combinatie voldoet aan alle eisen die gesteld zijn in de opdracht.
    De motor heeft een nominale snelheid van 3710 [rpm] en een nominale koppel van 28.7 [mNm].
    De gearbox heeft een vertanding van 5.2:1 en daardoor wordt de nominale snelheid 710 [rpm] en de nominale koppel van 150 [mNm].
    Dit is meer dan genoeg om een constante snelheid van 2.1 [m/s] te rijden op een helling van 20 graden.
    Ook kan er met deze motor en gearbox op een helling van 20 graden versneld worden met 0.7 [m/s] en vertraagd worden met 0.5 [m/s].
    Deze motor en gearbox combinatie is een van de meest efficiente combinaties die er zijn voor deze toepassing met de gestelde serie eisen.
    De beweerde efficientie van 90 procent is de hoogste in de RE25 1187 serie.
    Dit maakt hem volgens ons de beste keuze voor een rover die op zonne energie rijdt.


% All other content before appendices should go above this line
% Now, include the PDF files as appendices
% Place this right before the appendix command
\clearpage
\appendix
\section{Motor Datasheet}
\label{Bijlage A}
\includepdf[pages=-]{EN-22-152.pdf}
\section{Gearbox Datasheet}
\label{Bijlage B}
\includepdf[pages=-]{EN-21-390.pdf}
\section{Onderzoek naar maanlandschap NASA}
\label{Bijlage C}
\includepdf[pages=-]{Lunar_surface_models.pdf}

\end{document}