\documentclass{article}

\usepackage[english]{babel}
\usepackage{multicol}
\setlength{\columnsep}{1cm}

\usepackage[a4paper,top=2cm,bottom=2cm,left=2cm,right=2cm,marginparwidth=1.75cm]{geometry}

\usepackage{amsmath}
\usepackage{siunitx}
\usepackage{wrapfig}
\usepackage{float}
\usepackage{graphicx}
\usepackage{subcaption}
\usepackage[colorlinks=true, allcolors=blue]{hyperref}
\usepackage{xcolor}
\usepackage{lipsum}
\usepackage{mathtools}
\usepackage{listings}
\usepackage{xcolor}





\title{Aandrijftechniek maan casus}

\author{Jelmer Hemstra, 1810225, Flint Wardenaar, 1771881}


\begin{document}
\maketitle

\begin{abstract}
    In dit document wordt de casus van de aandrijftechniek van de maanlander behandeld. Hierbij wordt gekeken naar de verschillende aandrijftechnieken en de voor- en nadelen van deze technieken.
\end{abstract}



\section{Inleiding}
    In dit document wordt de casus van de aandrijftechniek van de maanlander behandeld. 


\section{Methodologie}
    Om te bepalen welk type motor het beste is voor de toepassing wordt er vooral gekeken naar de last die de motor moet verdragen.
    De last is opgedeelt in statische en dynamische last. 
    De statische last is de last die de motor moet verdragen als de maanlander een vaste snelheid heeft.
    De dynamische last is de last die de motor moet verdragen als de maanlander versnelt of vertraagt.
    \newline
    Beschrijf de aanpak en methoden die gebruikt zijn voor het onderzoek en de evaluatie van de motoren en transmissiesystemen.

\section{Analyse van de Lasteisen en -wensen}
    Om de last die de motor moet kunnen verdragen te bepalen wordt er gekeken naar de eisen die aan de maanlander worden gesteld.
    De maanlander moet een top snelheid kunnen halen van $2.1[m/s]$. 
    Ook moet hij kunnen versnellen met $0.7[m/s]$ en vertragen met $0.5[m/s]$. 
    Dit moet hij dan ook kunnen doen op een helling van $20^{\circ}$.
    \newline
    \newline
    Deze last is op te delen in een statische en dynamische last.
    De statische last is de last die de motor moet verdragen als de maanlander een vaste snelheid heeft.
    De dynamische last is de last die de motor moet verdragen als de maanlander versnelt of vertraagt.

\subsection{Statische last}

\subsection{Dynamische last}
\subsection{Krachten en Koppels}

\section{Mechanische Transmissie}
\subsection{Overbrengverhouding}
\subsection{Rendement van de Transmissie}

\section{Evaluatie van de Motor}
\subsection{Koppel en Snelheid}
\subsection{Efficiëntie van de Motor}
    Analyse van de door Maxon voorgestelde motor: RE25 1187xx, inclusief koppel, snelheid, en efficiëntie.

\section{Vergelijking en Alternatieve Opties}
    Eventuele alternatieve motoren en transmissiesystemen die overwogen zijn.

\section{Conclusie en Aanbevelingen}
    Samenvatting van de bevindingen en het uiteindelijke advies aan de HU over de geschiktheid van de Maxon motor en tandwieloverbrenging voor de Euro Moon Rover.

\appendix

\end{document}