\documentclass{article}

\usepackage[english]{babel}
\usepackage{multicol}
\setlength{\columnsep}{1cm}

\usepackage[a4paper,top=2cm,bottom=2cm,left=2cm,right=2cm,marginparwidth=1.75cm]{geometry}

\usepackage{amsmath}
\usepackage{siunitx}
\usepackage{wrapfig}
\usepackage{float}
\usepackage{graphicx}
\usepackage{subcaption}
\usepackage[colorlinks=true, allcolors=blue]{hyperref}
\usepackage{xcolor}
\usepackage{lipsum}
\usepackage{mathtools}
\usepackage{listings}
\usepackage{xcolor}





\title{Aandrijftechniek maan casus}

\author{Jelmer Hemstra, 1810225, Flint Wardenaar, 1771881}


\begin{document}
    \maketitle

    \begin{abstract}
        In dit document wordt de casus van de aandrijftechniek van de maanlander behandeld. Hierbij wordt gekeken naar de verschillende aandrijftechnieken en de voor- en nadelen van deze technieken.

    \end{abstract}


\section{Formules}
    $$
        F_{z2_{tot}} = F_z \cdot sin(\theta)
    $$

    Berekeningen
    Algemeen
    $$
        F_z [N] = m[kg] \cdot g[m/s]
    $$$$
        9.72 [N] = 6 [kg] \cdot 1.62[m/s^2]
    $$

    Per wiel
    Plat rijdend
    $$
        F_{r_{wiel}}[N]= \frac {R_r \cdot F_n[N]}{4}
    $$
    $$
        0.243 [N]=  \frac {0.1 \cdot 9.72[N]}{4}
    $$
    Slope 30 graden omhoog
    $$
        F_{n_{wielslope}} [N] = \frac {F_z [N] \cdot cos(\theta)}{4}
    $$
    $$
        0.21 [N] = \frac {9.72[N] \cdot cos(30)}{4}
    $$

\end{document}