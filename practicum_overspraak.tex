\documentclass{article}

\usepackage[english]{babel}
\usepackage{multicol}
\setlength{\columnsep}{1cm}

\usepackage[a4paper,top=2cm,bottom=2cm,left=2cm,right=2cm,marginparwidth=1.75cm]{geometry}

\usepackage{amsmath}
\usepackage{siunitx}
\usepackage{wrapfig}
\usepackage{float}
\usepackage{graphicx}
\usepackage{subcaption}
\usepackage[colorlinks=true, allcolors=blue]{hyperref}
\usepackage{xcolor}
\usepackage{lipsum}
\usepackage{mathtools}
\usepackage{listings}
\usepackage{xcolor}





\title{Practicum HARIMP: Overspraak}

\author{Jelmer Hemstra, 1810225, Flint Wardenaar, 1771881}


\begin{document}
    \maketitle

    \begin{abstract}
        In dit document wordt het practicum gedocumenteerd.

    \end{abstract}


    \section{Meetplan}
        \subsection{Opstelling}

            De testopstelling bestaat uit het een signaalgenerator die is aangesloten op  \(C_1\) en een oscilloscoop die is aangesloten op \(C_1\) en \(C_3\)/\(C_4\).
            Beide van deze connecties zijn gemaakt met behulp van een coax-kabel en een coax-splitter. 
            De afsluiting \(R_L\) kan worden verwisseld door 0, 220 en oneindig ohm. 
        \subsection{Plan}
        
            Op de signaalgenerator wordt door de frequenties gestapt met stappen van x10.    
            Door de peek-to-peek spanning op de uitgang af te lezen wordt gekeken bij welke frequenties de overspraak het grootst is.
            Deze frequentie wordt genoteerd als \(f_{max}\).
            
            De twee signalen worden d.m.v de oscilloscoop vergeleken.
            Voor elke afkap weerstand (0, 220, oneindig) wordt gekeken naar de volgende meetwaarden:
            \begin{itemize}
                \item Faseverschuiving \(X_N\)
                \item Amplitude \(X_N\)
                \item Faseverschuiving \(X_V\)
                \item Amplitude \(X_V\)
            \end{itemize}
            

        \subsection{Uitvoering}
            Na benadering blijkt dat \(f_{max} = 15 MHz\).
            Deze frequencie met een amplitude van 5 volt zijn de verschillende tests uitgevoerd.


            \begin{tabular}{|c|c|c|c|c|}
                \hline
                Afsluiting & Faseverschuiving \(X_N\) & Amplitude \(X_N\) & Faseverschuiving \(X_V\) & Amplitude \(X_V\) \\
                \hline
                \(\infty\) \(\Omega\) & 4\textdegree & 6 mV & 7\textdegree & 8 mV\\
                \hline
            \end{tabular}


    
            \section{Opdrachten}
                \subsection{Capacitieve overspraak}
                    \subsubsection*{A}
                        

\end{document}